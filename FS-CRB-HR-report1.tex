\documentclass[12pt,letterpaper,english,bibliography=totocnumbered,abstract=on]{scrartcl}

\usepackage{indentfirst}
\usepackage{appendix}
\usepackage{fullpage}
%\usepackage{subfiles}
\usepackage[T1]{fontenc}
\usepackage[latin9]{inputenc}
\usepackage{color}
\usepackage{babel}
\usepackage{verbatim}
\usepackage[unicode=true,pdfusetitle,
bookmarks=true,bookmarksnumbered=false,bookmarksopen=false,
breaklinks=true,pdfborder={0 0 0},pdfborderstyle={},backref=false,colorlinks=true]
{hyperref}
\hypersetup{linkcolor=blue,citecolor=blue,urlcolor=blue}

\usepackage{booktabs}
\usepackage{multirow}
\usepackage{adjustbox}
\usepackage{threeparttable}
\usepackage[table]{xcolor}
\usepackage{csquotes}
\usepackage{soul} % for hiliting text: \hl
% old style is authoryear
\usepackage[backend=biber, style=numeric, maxbibnames=99]{biblatex}
\addbibresource{mylibrary.bib}
\addbibresource{CRB.bib}

\usepackage[disable]{todonotes}

% Prevent page breaks within paragraphs
% https://tex.stackexchange.com/questions/21983/how-to-avoid-page-breaks-inside-paragraphs
\widowpenalties 1 10000


\begin{document}
\titlehead{US Forest Service Forest Health Protetion Grant Progess Report 1}
\title{Improving Coconut Rhinoceros Beetle Breeding Site Detection Using Harmonic Radar}
\author{Aubrey Moore, University of Guam}
%\date{Submitted December 30, 2020\\Revised January 28, 2021}
\maketitle
\begin{description}	
	\item[GRANTEE:] Aubrey Moore, University of Guam 
	\item[GRANT YEAR:] 2020
	\item[GRANT NUMBER:] 20-DG-11052021-227
	\item[GRANT PROGRAM:] Forest Health Protection
	\item[GRANT EXPIRATION DATE:] 2021-05-30
	\item[DATES COVERED BY THIS REPORT:] 2020-06-17 through 2020-12-31
	\item[GRANT STATUS:] Active
\end{description}	

\begin{footnotesize}
	\todo{change url}
\url{https://github.com/aubreymoore/FY19-PPA-Report-1/raw/master/PPA19-report2.pdf}
\end{footnotesize}


\newpage{}
\tableofcontents{}

\newpage
\listoftodos

\newpage


\section{OBJECTIVES AND SPECIFIC ACTIVITIES} 

%List the objectives individually that were included in the grant narrative.  Under those objectives include specific activities that occurred during the report timeframes.   

The objective of this grant project is to evaluate harmonic radar as an alternative to radio tracking for CRB breeding site detection. Detection of CRB breeding sites is essential for CRB control and eradication.

In a previous Forest Service feasability study, we successfully tracked CRB to cryptic breeding sites using miniature radio transmitters. However, the high cost of radio tags (>\$100 each) and short shelf life and field life of nonreplacable batteries (about 1 month and one week, respectively) preclude radio tracking for routine CRB breeding site detection make this technique too expensive for routine surveys. 

A promising alternative is harmonic radar technology. Harmonic radar tags do not require a battery. They are inexpensive (about \$1 each) and they have unlimited shelf-life and field life. We are evaluating hand-held harmonic radar equipment and tags manufactured by RECCO in Sweden. Rescuers use the hand-held units to rapidly locate victims wearing tags sown into their clothing. RECCO technology has been used to track other insects but has not yet been tried with CRB.

Our objective is not to track CRB tagged beetles in real-time, but to discover the end points of tags several days or even weeks after release of the tagged beetles.We anticipate that tags will accumulate at breeding sites.

Here is the plan from the \textit{verbatim} from the approved grant proposal:

\medskip
\colorbox[gray]{0.9}{\parbox{\textwidth}{
	\textbf{Schedule of Activities (2020)}
	\begin{description}
		\item[March-April:] Tag fabrication and testing at EMU.
		\item[April-early May:] Capture, flight test, and mark CRB at UoG.
		\item[May:] Conduct field releases and tracking of tagged CRB (two week intensive fieldwork period).
		\item[June-August:] Data analysis.
		\item[August-December:] Manuscript(s) and final report preparation.  Discuss findings with state agencies.  Make presentations at scientific meetings.  Plan further research with cooperators to implement findings in monitoring and control efforts.
	\end{description}
}}
\medskip

Progress to date includes:
\begin{itemize}
\item Procurement of harmonic radar equipment and tags
\item Tag fabrication (soldering antennae to diodes) in Matt Siderhurst's lab.
\item Preliminary field testing of equipment on Guam.
\end{itemize}

The most important activity is \textbf{Conduct field releases and tracking of tagged CRB (two week intensive fieldwork period)}. The intensive fieldwork will be performed on Guam in collaboration with Matt Siderhurst's team of students who have experience in tracking insects with harmonic radar (similar to what we did in the radio-tracking feasibility study). This activity has not yet been scheduled because of COVID-19 travel restrictions \ref{impediments}. Otherwise we are prepared to proceed.


\section{OUTPUTS} 

%Each Output listed in your grant narrative addressed individually.
%
%1) Output 1: list the output and what was accomplished.
%2) Output 2: list the output and what was accomplished.
%3) Etc.
%
%If output includes positions funded: (who, how long, type of work)
%
%If output includes Acres treated, include target species, number of acres and location 
%
%If output includes Acres surveyed, include target species, number of acres and location 
%
%Description and dollar value of equipment purchased, 
%
%Number of personnel trained. 

Nothing to report.

\section{MONITORING \& EVALUATION}

%If post-treatment monitoring has been completed, provide the results, especially results that show effectiveness of treatments.  
%
%List any project evaluations that took place to determine whether goals and Statewide Strategies are being met.

Nothing to report.

\section{BUDGET EXPENDITURES}

%Include budget table showing the expenditures (Personnel/salary costs, supplies purchased, contracts, etc.) that occurred during the reporting period

\begin{tabular}{lrrl}
	\hline
	Category & Budget & Spent & Note \\
	\hline 
	Equipment and supplies & \$8,000 & \$3,462 & Harmonic radar and tags \\ 
	Travel & \$12,000 & \$0 \\ 
	Admin. fee & \$3,000 & \$0 \\ 
	\hline 
\end{tabular} 

\label{impediments}
\section{PROBLEMS ENCOUNTERED THIS REPORTING PERIOD}

%Explain delays, adverse conditions or changed costs that significantly impair the ability to meet grant objectives.  If necessary, prepare a separate formal request for an extension of the grant period:

Progress on this project was impeded by COVID-19 travel restrictions which prevented collaborators from visiting Guam to participate in field work. Further delayed by delay was caused by Government of Guam \textit{stay at home} orders. The University of Guam was officially closed from March 20 to May 10 2020 and again from August 16 2020 to January 15 2021.

\section{CHANGES PLANNED}

% (If the scope of the objectives would change, or if more that 10\% of the total grant budget would change object class categories, prepare a separate formal request to amend the grant):

Nothing to report.

\section{CIVIL RIGHTS}

% - any activity that demonstrates Title VI compliance with civil rights requirements, such as documentation of outreach to underserved groups, public education documents in languages other than English, etc

Nothing to report.

\section{ATTACHMENTS}

% (photos of activities are very helpful, also electronic copies of survey and treatment maps, copies of developed brochures or posters, etc.):

None.

\section{PLANS}

% for work to be performed during next reporting period:

Nothing to report.

\end{document}
